% AMS-LaTeX Paper ************************************************
\documentclass{amsart}
\usepackage{graphicx}
\usepackage{enumitem}
\usepackage{amsmath}
\usepackage{amssymb}
\usepackage{array}
\usepackage{booktabs}
\usepackage{fancyhdr}

% ----------------------------------------------------------------
\vfuzz2pt % Don't report over-full v-boxes if over-edge is small
\hfuzz2pt % Don't report over-full h-boxes if over-edge is small

% THEOREMS -------------------------------------------------------
\newtheorem{thm}{Theorem}[section]
\newtheorem{cor}[thm]{Corollary}
\newtheorem{lem}[thm]{Lemma}
\newtheorem{prop}[thm]{Proposition}
\theoremstyle{definition}
\newtheorem{defn}[thm]{Definition}
% \theoremstyle{Exercise}
\newtheorem{ex}[thm]{Exercise}
\theoremstyle{remark}
\newtheorem{rem}[thm]{Remark}
% \theoremstyle{rule}
\newtheorem{rul}[thm]{Rule}

\numberwithin{equation}{section}

% MATH -----------------------------------------------------------
\newcommand{\norm}[1]{\left\Vert#1\right\Vert}
\newcommand{\abs}[1]{\left\vert#1\right\vert}
\newcommand{\set}[1]{\left\{#1\right\}}
\newcommand{\Real}{\mathbb R}
\newcommand{\Z}{\mathbb Z}
\newcommand{\To}{\longrightarrow}
\newcommand{\BX}{\bB(X)}
\newcommand{\A}{\mathcal{A}}
% ----------------------------------------------------------------

% define some simple, commonly-used commands
\newcommand{\eps}{\varepsilon}
\newcommand{\dsum}{\displaystyle\sum}
\newcommand{\dint}{\displaystyle\int}

\newcommand{\pdr}[2]{\dfrac{\partial{#1}}{\partial{#2}}}
\newcommand{\pdrr}[2]{\dfrac{\partial^2{#1}}{\partial{#2}^2}}
\newcommand{\pdrt}[3]{\dfrac{\partial^2{#1}}{\partial{#2}{\partial{#3}}}}
\newcommand{\dr}[2]{\dfrac{d{#1}}{d{#2}}}
\newcommand{\aver}[1]{\langle {#1} \rangle}
\newcommand{\Baver}[1]{\Big\langle {#1} \Big\rangle}

\newcommand{\bzero}{\mathbf 0}
\newcommand{\bGamma}{\mbox{\boldmath{$\Gamma$}}}
\newcommand{\btheta}{\boldsymbol \theta}
\newcommand{\bchi}{\mbox{\boldmath{$\chi$}}}
\newcommand{\bnu}{\boldsymbol \nu}
\newcommand{\bmu}{\boldsymbol \mu}
\newcommand{\brho}{\mbox{\boldmath{$\rho$}}}
\newcommand{\bxi}{\boldsymbol \xi}
\newcommand{\bnabla}{\boldsymbol \nabla}
\newcommand{\bOm}{\boldsymbol \Omega}
\newcommand{\blambda}{\boldsymbol \lambda}
\newcommand{\bsigma}{\boldsymbol \sigma}

\newcommand{\bbR}{\mathbb R}
\newcommand{\bbC}{\mathbb C}
\newcommand{\bbQ}{\mathbb Q}
\newcommand{\bbN}{\mathbb N}
\newcommand{\bbZ}{\mathbb Z}

\newcommand{\ba}{\mathbf a} \newcommand{\bb}{\mathbf b}
\newcommand{\bc}{\mathbf c} \newcommand{\bd}{\mathbf d}
\newcommand{\be}{\mathbf e} \newcommand{\bff}{\mathbf f}
\newcommand{\bg}{\mathbf g} \newcommand{\bh}{\mathbf h}
\newcommand{\bi}{\mathbf i} \newcommand{\bj}{\mathbf j}
\newcommand{\bk}{\mathbf k} \newcommand{\bl}{\mathbf l}
\newcommand{\bm}{\mathbf m} \newcommand{\bn}{\mathbf n}
\newcommand{\bo}{\mathbf o} \newcommand{\bp}{\mathbf p}
\newcommand{\bq}{\mathbf q} \newcommand{\br}{\mathbf r}
\newcommand{\bs}{\mathbf s} \newcommand{\bt}{\mathbf t}
\newcommand{\bu}{\mathbf u} \newcommand{\bv}{\mathbf v}
\newcommand{\bw}{\mathbf w} \newcommand{\bx}{\mathbf x}
\newcommand{\by}{\mathbf y} \newcommand{\bz}{\mathbf z}
\newcommand{\bA}{\mathbf A} \newcommand{\bB}{\mathbf B}
\newcommand{\bC}{\mathbf C} \newcommand{\bD}{\mathbf D}
\newcommand{\bE}{\mathbf E} \newcommand{\bF}{\mathbf F}
\newcommand{\bG}{\mathbf G} \newcommand{\bH}{\mathbf H}
\newcommand{\bI}{\mathbf I} \newcommand{\bJ}{\mathbf J}
\newcommand{\bK}{\mathbf K} \newcommand{\bL}{\mathbf L}
\newcommand{\bM}{\mathbf M} \newcommand{\bN}{\mathbf N}
\newcommand{\bO}{\mathbf O} \newcommand{\bP}{\mathbf P}
\newcommand{\bQ}{\mathbf Q} \newcommand{\bR}{\mathbf R}
\newcommand{\bS}{\mathbf S} \newcommand{\bT}{\mathbf T}
\newcommand{\bU}{\mathbf U} \newcommand{\bV}{\mathbf V}
\newcommand{\bW}{\mathbf W} \newcommand{\bX}{\mathbf X}
\newcommand{\bY}{\mathbf Y} \newcommand{\bZ}{\mathbf Z}

\newcommand{\cA}{\mathcal A} \newcommand{\cB}{\mathcal B}
\newcommand{\cC}{\mathcal C} \newcommand{\cD}{\mathcal D}
\newcommand{\cE}{\mathcal E} \newcommand{\cF}{\mathcal F}
\newcommand{\cG}{\mathcal G} \newcommand{\cH}{\mathcal H}
\newcommand{\cI}{\mathcal I} \newcommand{\cJ}{\mathcal J}
\newcommand{\cK}{\mathcal K} \newcommand{\cL}{\mathcal L}
\newcommand{\cM}{\mathcal M} \newcommand{\cN}{\mathcal N}
\newcommand{\cO}{\mathcal O} \newcommand{\cP}{\mathcal P}
\newcommand{\cQ}{\mathcal Q} \newcommand{\cR}{\mathcal R}
\newcommand{\cS}{\mathcal S} \newcommand{\cT}{\mathcal T}
\newcommand{\cU}{\mathcal U} \newcommand{\cV}{\mathcal V}
\newcommand{\cW}{\mathcal W} \newcommand{\cX}{\mathcal X}
\newcommand{\cY}{\mathcal Y} \newcommand{\cZ}{\mathcal Z}

%%%%%%%%%%%%%%Start%%%%%%%%%%%%%Start%%%%%%%%%%%Start%%%%%%%%%%%%%%%Start%%%%%%%%%%%%%%%%%%%%%%%%%Start%%%%%%%%%%%%%%%%
%%%%%%%%%%%%%%Start%%%%%%%%%%%%%Start%%%%%%%%%%%Start%%%%%%%%%%%%%%%Start%%%%%%%%%%%%%%%%%%%%%%%%%Start%%%%%%%%%%%%%%%%
%%%%%%%%%%%%%%Start%%%%%%%%%%%%%Start%%%%%%%%%%%Start%%%%%%%%%%%%%%%Start%%%%%%%%%%%%%%%%%%%%%%%%%Start%%%%%%%%%%%%%%%%
\usepackage{fancyhdr}

\pagestyle{fancy}
\fancyhf{}
\rhead{}
\chead{\includegraphics[scale=.1]{snhu_logo.png}}


\begin{document}
\title{Module Two Problem Set}

\begin{center}
\includegraphics[scale=.1]{snhu_logo.png}
\end{center}

%\thm{bbjh}
\maketitle
Banana \\\\\\\\\item Inverse error
\[
\begin{array}{||c||}
\hline \hline
p \to q\\
\neg p\\
\\
\begin{center}
%Enter your name below this line:
Ryan Hatch
\end{center}


\begin{center}
\rule{\textwidth}{0.4pt}
\end{center}


\newpage


\newpage

\section*{}
\section*{}

Directions: Type your solutions into this document and be sure to show all steps for arriving at your solution. Just giving a final number may not receive full credit.\\

\section*{Problem 1}
\subsection*{Part 1}
{\bf Indicate whether the argument is valid or invalid. For valid arguments, prove that the argument is valid using a truth table. For invalid arguments, give truth values for the variables showing that the argument is not valid.}\\
\begin{enumerate}

\item \[
\begin{array}{||c||}
\hline \hline
(p \land q) \to r\\
\\
\therefore (p \lor q) \to r\\
\hline \hline
\end{array}
\]\\\\
 %Enter your answer below this comment line.
\begin{table}[h]
\centering
\begin{tabular}{cccccc}
\toprule
P & Q & R & $(P \land Q) \rightarrow R$ & $(P \lor Q) \rightarrow R$ \\
\midrule
T & T & T & T & T \\
T & T & F & F & F \\
T & F & T & T & T \\
T & F & F & T & F \\
F & T & T & T & T \\
F & T & F & T & F \\
F & F & T & T & T \\
F & F & F & T & T \\
\bottomrule
\\
\end{tabular}
\\
\caption{Truth table: The argument is invalid as there are cases where the premise is true but the conclusion is false. For example, when P is true, Q is false, and R is false, as well as when P is false, Q is true, and R is false.}
\label{tab:truth_table_part1}
\end{table}
\end{enumerate}

\subsection*{Part 2}
{\bf Converse and inverse errors are typical forms of invalid arguments. Prove that each argument is invalid by giving truth values for the variables showing that the argument is invalid. You may find it easier to find the truth values by constructing a truth table.}\\
 \begin{enumerate}[label=(\alph*)]
\item Converse error
\[
\begin{array}{||c||}
\hline \hline
p \to q\\
q\\
\\
\therefore p\\
\hline \hline
\end{array}
\]\\\\
%Enter your answer below this comment line.
\subsection*{Converse Error}
The converse error argument is \textbf{Invalid} because there is a case where the premise is true (\( Q \)) but the conclusion is false (\( P \)), specifically in the third row.

\begin{table}[h]
\centering
\begin{tabular}{cccc}
\toprule
P & Q & Premise Q & Conclusion P (Converse Error) \\
\midrule
T & T & T & T \\
T & F & F & T \\
F & T & T & F \\
F & F & F & F \\
\bottomrule
\end{tabular}
\caption{Truth table for the Converse Error.}
\label{tab:converse_error}
\end{table}
 \\\\
\item Inverse error
\[
\begin{array}{||c||}
\hline \hline
p \to q\\
\neg p\\
\\
\therefore \neg q\\
\hline \hline
\end{array}
\]\\\\
%Enter your answer below this comment line.
\subsection*{Inverse Error}
The inverse error argument is \textbf{Invalid} because there are cases where the premise is true (not \( P \) is true) but the conclusion is false (not \( Q \) is false), specifically in the third row.

\begin{table}[h]
\centering
\begin{tabular}{cccc}
\toprule
P & Q & Premise \(\neg P\) & Conclusion \(\neg Q\) (Inverse Error) \\
\midrule
T & T & F & F \\
T & F & F & T \\
F & T & T & F \\
F & F & T & T \\
\bottomrule
\end{tabular}
\caption{Truth table for the Inverse Error.}
\label{tab:inverse_error}
\end{table}
\\\\
\end{enumerate}

\subsection*{Part 3}
{\bf Which of the following arguments are invalid and which are valid? Prove your answer by replacing each proposition with a variable to obtain the form of the argument. Then prove that the form is valid or invalid.}\\
 \begin{enumerate}[label=(\alph*)]
  \item \[
\begin{array}{||c||}
\hline \hline
\text{The patient has high blood pressure or diabetes or both.}\\
\text {The patient has diabetes or high cholesterol or both.}\\
\\
\therefore \text {The patient has high blood pressure or high cholesterol.
}\\
\hline \hline
\end{array}
\]\\\\\
%Enter your answer below this comment line.
\begin{table}[h]
\centering
\begin{tabular}{cccccc}
\toprule
H (Blood Pressure) & D (Diabetes) & C (Cholesterol) & Combined Premise & Conclusion \\
\midrule
T & T & T & T & T \\
T & T & F & T & T \\
T & F & T & T & T \\
T & F & F & F & T \\
F & T & T & T & T \\
F & T & F & T & F \\
F & F & T & F & T \\
F & F & F & F & F \\
\bottomrule
\end{tabular}
\caption{The argument is invalid as there is at least one case where the combined premise is true but the conclusion is false. For example, when the patient has diabetes (D is true) but neither high blood pressure (H is false) nor high cholesterol (C is false), as shown in the sixth row.}
\label{tab:truth_table_part3}
\end{table}
\\\\\

 
    \end{enumerate}
 \newpage
%--------------------------------------------------------------------------------------------------

\section*{Problem 2}
\subsection*{Part 1}

 Which of the following arguments are valid? Explain your reasoning.\\
 \begin{enumerate}[label=(\alph*)]
\item I have a student in my class who is getting an $A$. Therefore, John, a student in my class, is getting an $A$. \\\\
%Enter your answer below this comment line.
This argument is invalid. It commits the fallacy of division: assuming that what is true for the group as a whole must be true for any individual member.
\\\\
\item Every Girl Scout who sells at least 30 boxes of cookies will get a prize. Suzy, a Girl Scout, got a prize. Therefore, Suzy sold at least 30 boxes of cookies.\\\\
%Enter your answer below this comment line.
This argument is invalid. It commits the fallacy of affirming that just because Suzy received a prize, which is one possible consequence of selling 30 boxes, it does not necessarily mean that was the condition met to receive the prize.
\\\\
 \end{enumerate}

 \subsection*{Part 2}
Determine whether each argument is valid. If the argument is valid, give a proof using the laws of logic. If the argument is invalid, give values for the predicates $P$ and $Q$ over the domain ${a,\; b}$ that demonstrate the argument is invalid.\\
 \begin{enumerate}[label=(\alph*)]
\item \[
\begin{array}{||c||}
\hline \hline
\exists x\, (P(x)\; \land \;Q(x) )\\
\\
\therefore \exists x\, Q(x)\; \land\; \exists x \,P(x) \\
\hline \hline
\end{array}
\]\\\\
 %Enter your answer here.
This argument is valid. If there is some \( x \) in the domain for which both \( P(x) \) and \( Q(x) \) are true, then there must exist an \( x \) for which \( Q(x) \) is true and an \( x \) (not necessarily the same) for which \( P(x) \) is true.
 \\\\


\item \[
\begin{array}{||c||}
\hline \hline
\forall x\, (P(x)\; \lor \;Q(x) )\\
\\
\therefore \forall x\, Q(x)\; \lor \; \forall x\, P(x) \\
\hline \hline
\end{array}
\]\\\\
 %Enter your answer here.
This argument is invalid. As a counterexample, the domain will be \( \{a, b\} \) with \( P(a) \) true and \( Q(b) \) true. Therefor,\( \forall x (P(x) \lor Q(x)) \) is true, but neither \( \forall x Q(x) \) nor \( \forall x P(x) \) is true since no single property holds for all elements of the domain.
 \\\\
 \end{enumerate}
 \newpage
%--------------------------------------------------------------------------------------------------


\section*{Problem 3}

Prove the following using a direct proof. Your proof should be expressed in complete English sentences.
\\\\

If $a$, $b$, and $c$ are integers such that $b$ is a multiple of $a^3$ and $c$ is a multiple of $b^2$, then $c$ is a multiple of $a^6$.
\\\\
%Enter your answer below this comment line.
\paragraph{Proof:} Let \( a, b, \) and \( c \) be integers with the given properties. Since \( b \) is a multiple of \( a^3 \), there exists an integer \( k \) such that \( b = a^3k \). Since \( c \) is a multiple of \( b^2 \), there exists an integer \( m \) such that \( c = b^2m \).

Substituting \( b \) into the expression for \( c \) I got
\begin{align*}
c &= (a^3k)^2m \\
  &= a^6k^2m.
\end{align*}
Since the product of integers is an integer, \( k^2m \) is an integer. There fore, expressing \( c \) as \( a^6 \) times an integer, which proved that \( c \) is a multiple of \( a^6 \).

\(\square\)
\\\\

 \newpage
%--------------------------------------------------------------------------------------------------
\section*{Problem 4}
Prove the following using a direct proof:
\\

The sum of the squares of 4 consecutive integers is an even integer.
%Enter your answer below this comment line.
\paragraph{Proof:} If the first integer is \(n\), it makes the four consecutive integers \(n\), \(n+1\), \(n+2\), and \(n+3\). The sum of the squares of these integers is:

\begin{align*}
n^2 + (n+1)^2 + (n+2)^2 + (n+3)^2 &= n^2 + n^2 + 2n + 1 + n^2 + 4n + 4 + n^2 + 6n + 9 \\
&= 4n^2 + 12n + 14 \\
&= 2(2n^2 + 6n + 7)
\end{align*}

Since \(2n^2 + 6n + 7\) is an integer, the entire expression is even because it is two times some integer. Therefore, the sum of the squares of 4 consecutive integers is an even integer.

\(\square\)
\\\\


 \newpage
%--------------------------------------------------------------------------------------------------
\section*{Problem 5}

Prove the following using a proof by contrapositive:
\\\\

Let $x$ be a rational number. Prove that if $xy$ is irrational, then y is irrational.\\\\
%Enter your answer below this comment line.
If \( y \) were rational, then \( xy \) would also be rational because the product of two rational numbers is rational. Since \( xy \) is given to be irrational, \( y \) must be irrational.
\\\\





 \newpage
%--------------------------------------------------------------------------------------------------

\section*{Problem 6}
Prove the following using a proof by contradiction:
\\\\


The average of four real numbers is greater than or equal to at least one of the numbers.
%Enter your answer below this comment line.
Assuming for contradiction that the average is less than all four numbers. Adding all four inequalities would then give a sum greater than four times the average, which contradicts the definition of an average. Which means, at least one number must be less than or equal to the average.
\\\\



 \newpage
%--------------------------------------------------------------------------------------------------

\section*{Problem 7}

Let $\displaystyle q = \frac{a}{b}$ and $\displaystyle r = \frac{c}{d}$ be two rational numbers written in lowest terms. Let $s = q + r$ and $\displaystyle s = \frac{e}{f}$ be written in lowest terms. Assume that $s$ is not $0$.\\

 Prove or disprove the following two statements.
\\\\
a.  If $b$ and $d$ are odd, then $f$ is odd.
\\\\
b. If $b$ and $d$ are even, then $f$ is even.
\\\\
%Enter your answer below this comment line.
This statement is not exactly true. If \( b \) and \( d \) are even, \( f \) could still be odd if the numerator and denominator of \( s \) share a common even factor, which is then canceled out to give \( s \) in lowest terms. For example, \( \frac{2}{4} + \frac{2}{4} = \frac{4}{4} = 1 \), where \( b = d = 4 \) and \( f = 1 \).
\\\\


\newpage

\section*{Problem 8}
{\bf Define $P(n)$ to be the assertion that:}\\
\[\displaystyle \sum_{j=1}^{n}\, j^2 \;=\;\frac{n(n+1)(2n+1)}{6}\]\\\\
\begin{enumerate}[label=(\alph*)]
  \item Verify that $P(3)$ is true.\\\\
   %Enter your answer here.
   First, I calculated both sides for $n=3$: \\
   \[\text{Left Hand Side} = 1^2 + 2^2 + 3^2 = 1 + 4 + 9 = 14\] \\
   \[\text{Right Hand Side} = \frac{3(3+1)(2 \cdot 3+1)}{6} = \frac{3 \cdot 4 \cdot 7}{6} = 14\] \\
   Since the Left Hand Side = the Right Hand Side, $P(3)$ is true. \\\\
  \item Express $P(k)$.\\\\
   $P(k)$ is expressed as: \\
   \[\sum_{j=1}^{k}\, j^2 = \frac{k(k+1)(2k+1)}{6}\] \\\\
   \\\\
  \item Express $P(k)$.\\\\
   %Enter your answer here.
   $P(k)$ is expressed as: \\
   \[\sum_{j=1}^{k}\, j^2 = \frac{k(k+1)(2k+1)}{6}\] \\\\
   \\\\
  \item Express $P(k+1)$.\\\\
   %Enter your answer here.
   $P(k+1)$ is expressed as: \\
   \[\sum_{j=1}^{k+1}\, j^2 = \frac{(k+1)(k+2)(2(k+1)+1)}{6}\] \\\\
   \item In an inductive proof that for every positive integer $n$,
   \\\\
   \[\displaystyle \sum_{j=1}^{n}\, j^2 \;=\;\frac{n(n+1)(2n+1)}{6}\]
   what must be proven in the base case?\\\\
    %Enter your answer here.
    In the base case, you have to prove that the statement holds for $n=1$. \\\\
    \item In an inductive proof that for every positive integer $n$,
    \\\\
    \item In an inductive proof that for every positive integer $n$,
   \[\displaystyle \sum_{j=1}^{n}\, j^2 \;=\;\frac{n(n+1)(2n+1)}{6}\]
   what must be proven in the inductive step?\\\\
   %Enter your answer here.
   In the inductive step, you prove that if $P(k)$ is true, then $P(k+1)$ is also true. \\\\
   \\\\
   \item What would be the inductive hypothesis in the inductive step from your previous answer?\\\\
    %Enter your answer here.
    The inductive hypothesis would be the assumption that $P(k)$ is true for some arbitrary positive integer $k$. \\\\
    \\\\
   \item Prove by induction that for any positive integer n,
   \[\displaystyle \sum_{j=1}^{n}\, j^2 \;=\;\frac{n(n+1)(2n+1)}{6}\] \\\\
    %Enter your answer here.
    The proof by induction would involve verifying the base case $P(1)$, assuming the inductive hypothesis $P(k)$, and then proving $P(k+1)$ based on that assumption. \\\\
\end{enumerate}

\end{document}