% AMS-LaTeX Paper ************************************************
\documentclass{amsart}
\usepackage{graphicx}
\usepackage{enumitem}
\usepackage{amsmath}
\usepackage{amssymb}
\usepackage{array}
\usepackage{booktabs}
\usepackage{fancyhdr}

% ----------------------------------------------------------------
\vfuzz2pt % Don't report over-full v-boxes if over-edge is small
\hfuzz2pt % Don't report over-full h-boxes if over-edge is small

% THEOREMS -------------------------------------------------------
\newtheorem{thm}{Theorem}[section]
\newtheorem{cor}[thm]{Corollary}
\newtheorem{lem}[thm]{Lemma}
\newtheorem{prop}[thm]{Proposition}
\theoremstyle{definition}
\newtheorem{defn}[thm]{Definition}
\newtheorem{ex}[thm]{Exercise}
\theoremstyle{remark}
\newtheorem{rem}[thm]{Remark}
\newtheorem{rul}[thm]{Rule}

\numberwithin{equation}{section}

% MATH -----------------------------------------------------------
\newcommand{\norm}[1]{\left\Vert#1\right\Vert}
\newcommand{\abs}[1]{\left\vert#1\right\vert}
\newcommand{\set}[1]{\left\{#1\right\}}
\newcommand{\Real}{\mathbb R}
\newcommand{\Z}{\mathbb Z}
\newcommand{\To}{\longrightarrow}
\newcommand{\BX}{\bB(X)}
\newcommand{\A}{\mathcal{A}}

% ----------------------------------------------------------------

% define some simple, commonly-used commands
\newcommand{\eps}{\varepsilon}
\newcommand{\dsum}{\displaystyle\sum}
\newcommand{\dint}{\displaystyle\int}

\newcommand{\pdr}[2]{\dfrac{\partial{#1}}{\partial{#2}}}
\newcommand{\pdrr}[2]{\dfrac{\partial^2{#1}}{\partial{#2}^2}}
\newcommand{\pdrt}[3]{\dfrac{\partial^2{#1}}{\partial{#2}{\partial{#3}}}}
\newcommand{\dr}[2]{\dfrac{d{#1}}{d{#2}}}
\newcommand{\aver}[1]{\langle {#1} \rangle}
\newcommand{\Baver}[1]{\Big\langle {#1} \Big\rangle}

\newcommand{\bzero}{\mathbf 0}
\newcommand{\bGamma}{\mbox{\boldmath{$\Gamma$}}}
\newcommand{\btheta}{\boldsymbol \theta}
\newcommand{\bchi}{\mbox{\boldmath{$\chi$}}}
\newcommand{\bnu}{\boldsymbol \nu}
\newcommand{\bmu}{\boldsymbol \mu}
\newcommand{\brho}{\mbox{\boldmath{$\rho$}}}
\newcommand{\bxi}{\boldsymbol \xi}
\newcommand{\bnabla}{\boldsymbol \nabla}
\newcommand{\bOm}{\boldsymbol \Omega}
\newcommand{\blambda}{\boldsymbol \lambda}
\newcommand{\bsigma}{\boldsymbol \sigma}

\newcommand{\bbR}{\mathbb R}
\newcommand{\bbC}{\mathbb C}
\newcommand{\bbQ}{\mathbb Q}
\newcommand{\bbN}{\mathbb N}
\newcommand{\bbZ}{\mathbb Z}

\newcommand{\ba}{\mathbf a} \newcommand{\bb}{\mathbf b}
\newcommand{\bc}{\mathbf c} \newcommand{\bd}{\mathbf d}
\newcommand{\be}{\mathbf e} \newcommand{\bff}{\mathbf f}
\newcommand{\bg}{\mathbf g} \newcommand{\bh}{\mathbf h}
\newcommand{\bi}{\mathbf i} \newcommand{\bj}{\mathbf j}
\newcommand{\bk}{\mathbf k} \newcommand{\bl}{\mathbf l}
\newcommand{\bm}{\mathbf m} \newcommand{\bn}{\mathbf n}
\newcommand{\bo}{\mathbf o} \newcommand{\bp}{\mathbf p}
\newcommand{\bq}{\mathbf q} \newcommand{\br}{\mathbf r}
\newcommand{\bs}{\mathbf s} \newcommand{\bt}{\mathbf t}
\newcommand{\bu}{\mathbf u} \newcommand{\bv}{\mathbf v}
\newcommand{\bw}{\mathbf w} \newcommand{\bx}{\mathbf x}
\newcommand{\by}{\mathbf y} \newcommand{\bz}{\mathbf z}
\newcommand{\bA}{\mathbf A} \newcommand{\bB}{\mathbf B}
\newcommand{\bC}{\mathbf C} \newcommand{\bD}{\mathbf D}
\newcommand{\bE}{\mathbf E} \newcommand{\bF}{\mathbf F}
\newcommand{\bG}{\mathbf G} \newcommand{\bH}{\mathbf H}
\newcommand{\bI}{\mathbf I} \newcommand{\bJ}{\mathbf J}
\newcommand{\bK}{\mathbf K} \newcommand{\bL}{\mathbf L}
\newcommand{\bM}{\mathbf M} \newcommand{\bN}{\mathbf N}
\newcommand{\bO}{\mathbf O} \newcommand{\bP}{\mathbf P}
\newcommand{\bQ}{\mathbf Q} \newcommand{\bR}{\mathbf R}
\newcommand{\bS}{\mathbf S} \newcommand{\bT}{\mathbf T}
\newcommand{\bU}{\mathbf U} \newcommand{\bV}{\mathbf V}
\newcommand{\bW}{\mathbf W} \newcommand{\bX}{\mathbf X}
\newcommand{\bY}{\mathbf Y} \newcommand{\bZ}{\mathbf Z}

\newcommand{\cA}{\mathcal A} \newcommand{\cB}{\mathcal B}
\newcommand{\cC}{\mathcal C} \newcommand{\cD}{\mathcal D}
\newcommand{\cE}{\mathcal E} \newcommand{\cF}{\mathcal F}
\newcommand{\cG}{\mathcal G} \newcommand{\cH}{\mathcal H}
\newcommand{\cI}{\mathcal I} \newcommand{\cJ}{\mathcal J}
\newcommand{\cK}{\mathcal K} \newcommand{\cL}{\mathcal L}
\newcommand{\cM}{\mathcal M} \newcommand{\cN}{\mathcal N}
\newcommand{\cO}{\mathcal O} \newcommand{\cP}{\mathcal P}
\newcommand{\cQ}{\mathcal Q} \newcommand{\cR}{\mathcal R}
\newcommand{\cS}{\mathcal S} \newcommand{\cT}{\mathcal T}
\newcommand{\cU}{\mathcal U} \newcommand{\cV}{\mathcal V}
\newcommand{\cW}{\mathcal W} \newcommand{\cX}{\mathcal X}
\newcommand{\cY}{\mathcal Y} \newcommand{\cZ}{\mathcal Z}

\pagestyle{fancy}
\fancyhf{}
\rhead{}
\chead{\includegraphics[scale=.1]{snhu_logo.png}}

\begin{document}
\title{Module Two Problem Set}

\maketitle
\includegraphics[scale=.1]{snhu_logo.png}

\section*{Problem 1}
\subsection*{Part 1}
\textbf{Indicate whether the argument is valid or invalid. For valid arguments, prove that the argument is valid using a truth table. For invalid arguments, give truth values for the variables showing that the argument is not valid.}
\begin{enumerate}
    \item 
    \[
    \begin{array}{||c||}
    \hline \hline
    (p \land q) \to r \\
    \therefore (p \lor q) \to r \\
    \hline \hline
    \end{array}
    \]
    \begin{table}[h]
    \centering
    \begin{tabular}{cccccc}
    \toprule
    P & Q & R & $(P \land Q) \rightarrow R$ & $(P \lor Q) \rightarrow R$ \\
    \midrule
    T & T & T & T                           & T                          \\
    T & T & F & F                           & F                          \\
    T & F & T & T                           & T                          \\
    T & F & F & T                           & F                          \\
    F & T & T & T                           & T                          \\
    F & T & F & T                           & F                          \\
    F & F & T & T                           & T                          \\
    F & F & F & T                           & T                          \\
    \bottomrule
    \end{tabular}
    \caption{Truth table: The argument is invalid as there are cases where the premise is true but the conclusion is false. For example, when P is true, Q is false, and R is false, as well as when P is false, Q is true, and R is false.}
    \label{tab:truth_table_part1}
    \end{table}
\end{enumerate}

\section*{Problem 2}
\subsection*{Part 1}
\textbf{Which of the following arguments are valid? Explain your reasoning.}
\begin{enumerate}[label=(\alph*)]
    \item I have a student in my class who is getting an $A$. Therefore, John, a student in my class, is getting an $A$.
    This argument is invalid. It commits the fallacy of division, assuming what is true for the whole (the class) must be true for each individual member (John).

    \item Every Girl Scout who sells at least 30 boxes of cookies will get a prize. Suzy, a Girl Scout, got a prize. Therefore, Suzy sold at least 30 boxes of cookies.
    This argument is invalid. It commits the fallacy of affirming the consequent. Just because Suzy got a prize, it does not necessarily mean she met the specific condition of selling 30 boxes.
\end{enumerate}

\subsection*{Part 2}
\textbf{Determine whether each argument is valid. If the argument is valid, give a proof using the laws of logic. If the argument is invalid, give values for the predicates $P$ and $Q$ over the domain ${a,\; b}$ that demonstrate the argument is invalid.}
\begin{enumerate}[label=(\alph*)]
    \item 
    % Argument and proof or disproof.
    \item 
    % Argument and proof or disproof.
\end{enumerate}

% ...

\section*{Problem 3}
\textbf{Prove the following using a direct proof. Your proof should be expressed in complete English sentences.}
\begin{enumerate}
    \item If $a$, $b$, and $c$ are integers such that $b$ is a multiple of $a^3$ and $c$ is a multiple of $b^2$, then $c$ is a multiple of $a^6$.
    %Proof: ...
\end{enumerate}

% ...

\section*{Problem 4}
\textbf{Prove the following using a direct proof:}
\begin{enumerate}
    \item The sum of the squares of 4 consecutive integers is an even integer.
    %Proof: ...
\end{enumerate}

% ...

\section*{Problem 5}
\textbf{Prove the following using a proof by contrapositive:}
\begin{enumerate}
    \item Let $x$ be a rational number. Prove that if $xy$ is irrational, then $y$ is irrational.
    %Proof: ...
\end{enumerate}

% ...

\section*{Problem 6}
\textbf{Prove the following using a proof by contradiction:}
\begin{enumerate}
    \item The average of four real numbers is greater than or equal to at least one of the numbers.
    %Proof: ...
\end{enumerate}

% ...

\section*{Problem 7}
\textbf{Let $\displaystyle q = \frac{a}{b}$ and $\displaystyle r = \frac{c}{d}$ be two rational numbers written in lowest terms. Let $s = q + r$ and $\displaystyle s = \frac{e}{f}$ be written in lowest terms. Assume that $s$ is not $0$. Prove or disprove the following two statements.}
\begin{enumerate}[label=(\alph*)]
    \item If $b$ and $d$ are odd, then $f$ is odd.
    %Proof or disproof: ...
    \item If $b$ and $d$ are even, then $f$ is even.
    %Proof or disproof: ...
\end{enumerate}

% ...

\section*{Problem 8}
\textbf{Define $P(n)$ to be the assertion that:}
\[\displaystyle \sum_{j=1}^{n}\, j^2 = \frac{n(n+1)(2n+1)}{6}\]
\begin{enumerate}[label=(\alph*)]
    \item Verify that $P(3)$ is true.
    %Verification: ...
    \item Express $P(k)$.
    %Expression: ...
    \item Express $P(k+1)$.
    %Expression: ...
    \item In an inductive proof that for every positive integer $n$, prove that \[\displaystyle \sum_{j=1}^{n}\, j^2 = \frac{n(n+1)(2n+1)}{6}\] is true.
    %Inductive proof: ...
\end{enumerate}

\end{document}
