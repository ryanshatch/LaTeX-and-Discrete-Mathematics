% ----------------------------------------------------------------
% AMS-LaTeX Paper ************************************************
% **** -----------------------------------------------------------
%\documentclass{amsart}
%\usepackage{txfonts}
%\documentclass[12pt,oneside]{article}
\documentclass{amsart}
\usepackage{graphicx}
\usepackage{enumitem}
% ----------------------------------------------------------------
\vfuzz2pt % Don't report over-full v-boxes if over-edge is small
\hfuzz2pt % Don't report over-full h-boxes if over-edge is small
% THEOREMS -------------------------------------------------------
\newtheorem{thm}{Theorem}[section]
\newtheorem{cor}[thm]{Corollary}
\newtheorem{lem}[thm]{Lemma}
\newtheorem{prop}[thm]{Proposition}
\theoremstyle{definition}
\newtheorem{defn}[thm]{Definition}
\theoremstyle{Exercise}
\newtheorem{ex}[thm]{Exercise}
\theoremstyle{remark}
\newtheorem{rem}[thm]{Remark}
\theoremstyle{rule}
\newtheorem{rul}[thm]{Rule}

\numberwithin{equation}{section}
% MATH -----------------------------------------------------------
\newcommand{\norm}[1]{\left\Vert#1\right\Vert}
\newcommand{\abs}[1]{\left\vert#1\right\vert}
\newcommand{\set}[1]{\left\{#1\right\}}
\newcommand{\Real}{\mathbb R}
\newcommand{\Z}{\mathbb Z}
\newcommand{\To}{\longrightarrow}
\newcommand{\BX}{\bB(X)}
\newcommand{\A}{\mathcal{A}}
% ----------------------------------------------------------------

% define some simple, commonly-used commands
\newcommand{\eps}{\varepsilon}
\newcommand{\dsum}{\displaystyle\sum}
\newcommand{\dint}{\displaystyle\int}

\newcommand{\pdr}[2]{\dfrac{\partial{#1}}{\partial{#2}}}
\newcommand{\pdrr}[2]{\dfrac{\partial^2{#1}}{\partial{#2}^2}}
\newcommand{\pdrt}[3]{\dfrac{\partial^2{#1}}{\partial{#2}{\partial{#3}}}}
\newcommand{\dr}[2]{\dfrac{d{#1}}{d{#2}}}
\newcommand{\aver}[1]{\langle {#1} \rangle}
\newcommand{\Baver}[1]{\Big\langle {#1} \Big\rangle}

\newcommand{\bzero}{\mathbf 0}
\newcommand{\bGamma}{\mbox{\boldmath{$\Gamma$}}}
\newcommand{\btheta}{\boldsymbol \theta}
\newcommand{\bchi}{\mbox{\boldmath{$\chi$}}}
\newcommand{\bnu}{\boldsymbol \nu}
\newcommand{\bmu}{\boldsymbol \mu}
\newcommand{\brho}{\mbox{\boldmath{$\rho$}}}
\newcommand{\bxi}{\boldsymbol \xi}
\newcommand{\bnabla}{\boldsymbol \nabla}
\newcommand{\bOm}{\boldsymbol \Omega}
\newcommand{\blambda}{\boldsymbol \lambda}
\newcommand{\bsigma}{\boldsymbol \sigma}

\newcommand{\bbR}{\mathbb R}
\newcommand{\bbC}{\mathbb C}
\newcommand{\bbQ}{\mathbb Q}
\newcommand{\bbN}{\mathbb N}
\newcommand{\bbZ}{\mathbb Z}

\newcommand{\ba}{\mathbf a} \newcommand{\bb}{\mathbf b}
\newcommand{\bc}{\mathbf c} \newcommand{\bd}{\mathbf d}
\newcommand{\be}{\mathbf e} \newcommand{\bff}{\mathbf f}
\newcommand{\bg}{\mathbf g} \newcommand{\bh}{\mathbf h}
\newcommand{\bi}{\mathbf i} \newcommand{\bj}{\mathbf j}
\newcommand{\bk}{\mathbf k} \newcommand{\bl}{\mathbf l}
\newcommand{\bm}{\mathbf m} \newcommand{\bn}{\mathbf n}
\newcommand{\bo}{\mathbf o} \newcommand{\bp}{\mathbf p}
\newcommand{\bq}{\mathbf q} \newcommand{\br}{\mathbf r}
\newcommand{\bs}{\mathbf s} \newcommand{\bt}{\mathbf t}
\newcommand{\bu}{\mathbf u} \newcommand{\bv}{\mathbf v}
\newcommand{\bw}{\mathbf w} \newcommand{\bx}{\mathbf x}
\newcommand{\by}{\mathbf y} \newcommand{\bz}{\mathbf z}
\newcommand{\bA}{\mathbf A} \newcommand{\bB}{\mathbf B}
\newcommand{\bC}{\mathbf C} \newcommand{\bD}{\mathbf D}
\newcommand{\bE}{\mathbf E} \newcommand{\bF}{\mathbf F}
\newcommand{\bG}{\mathbf G} \newcommand{\bH}{\mathbf H}
\newcommand{\bI}{\mathbf I} \newcommand{\bJ}{\mathbf J}
\newcommand{\bK}{\mathbf K} \newcommand{\bL}{\mathbf L}
\newcommand{\bM}{\mathbf M} \newcommand{\bN}{\mathbf N}
\newcommand{\bO}{\mathbf O} \newcommand{\bP}{\mathbf P}
\newcommand{\bQ}{\mathbf Q} \newcommand{\bR}{\mathbf R}
\newcommand{\bS}{\mathbf S} \newcommand{\bT}{\mathbf T}
\newcommand{\bU}{\mathbf U} \newcommand{\bV}{\mathbf V}
\newcommand{\bW}{\mathbf W} \newcommand{\bX}{\mathbf X}
\newcommand{\bY}{\mathbf Y} \newcommand{\bZ}{\mathbf Z}

\newcommand{\cA}{\mathcal A} \newcommand{\cB}{\mathcal B}
\newcommand{\cC}{\mathcal C} \newcommand{\cD}{\mathcal D}
\newcommand{\cE}{\mathcal E} \newcommand{\cF}{\mathcal F}
\newcommand{\cG}{\mathcal G} \newcommand{\cH}{\mathcal H}
\newcommand{\cI}{\mathcal I} \newcommand{\cJ}{\mathcal J}
\newcommand{\cK}{\mathcal K} \newcommand{\cL}{\mathcal L}
\newcommand{\cM}{\mathcal M} \newcommand{\cN}{\mathcal N}
\newcommand{\cO}{\mathcal O} \newcommand{\cP}{\mathcal P}
\newcommand{\cQ}{\mathcal Q} \newcommand{\cR}{\mathcal R}
\newcommand{\cS}{\mathcal S} \newcommand{\cT}{\mathcal T}
\newcommand{\cU}{\mathcal U} \newcommand{\cV}{\mathcal V}
\newcommand{\cW}{\mathcal W} \newcommand{\cX}{\mathcal X}
\newcommand{\cY}{\mathcal Y} \newcommand{\cZ}{\mathcal Z}


%%%%%%%%%%%%%%Start%%%%%%%%%%%%%Start%%%%%%%%%%%Start%%%%%%%%%%%%%%%Start%%%%%%%%%%%%%%%%%%%%%%%%%Start%%%%%%%%%%%%%%%%
%%%%%%%%%%%%%%Start%%%%%%%%%%%%%Start%%%%%%%%%%%Start%%%%%%%%%%%%%%%Start%%%%%%%%%%%%%%%%%%%%%%%%%Start%%%%%%%%%%%%%%%%
%%%%%%%%%%%%%%Start%%%%%%%%%%%%%Start%%%%%%%%%%%Start%%%%%%%%%%%%%%%Start%%%%%%%%%%%%%%%%%%%%%%%%%Start%%%%%%%%%%%%%%%%

\usepackage{fancyhdr}

\pagestyle{fancy}
\fancyhf{}
\rhead{}
\chead{\includegraphics[scale=.1]{logo.png}}

\begin{document}
\begin{center}
\includegraphics[scale=.1]{logo.png}
\end{center}
\title{\sf Discrete Mathematics}%


%\thm{bbjh}
\maketitle
Counting and Probability
\begin{center}
%Enter your name below this line:
Ryan Hatch
\end{center}


\begin{center}
\rule{\textwidth}{0.4pt}
\end{center}


\newpage
\section*{}
\section*{}
Directions: Type your solutions into this document and be sure to show all steps for arriving at your solution. Just giving a final number may not receive full credit.
\\\\



%--------------------------------------------------------------------------------------------------

\section*{Problem 1}

A 125-page document is being printed by five printers. Each page will be printed exactly once.
 \begin{enumerate}[label=(\alph*)]
 \item  Suppose that there are no restrictions on how many pages a printer can print. How many ways are there for the 125 pages to be assigned to the five printers?\\\\
{\it One possible combination: printer A prints out pages 2-50, printer B prints out pages 1 and 51-60, printer C prints out 61-80 and 86-90, printer D prints out pages 81-85 and 91-100, and printer E prints out pages 101-125.}\\\\
%Enter your answer below this comment line.
--------------------------------------------------------------------------------------------------\\
Considering that each of the 125 pages can be assigned3d to any of the 5 printers independently,  this would make the total number of ways to assign the pages to be \(5^{125}\). For each page, since there are 5 printers to choose from, and also since there are 125 pages, the choices multiply, which gave me \(5^{125}\) total ways.\\
--------------------------------------------------------------------------------------------------\\
        \\\\
 \item Suppose the first and the last page of the document must be printed in color, and only two printers are able to print in color. The two color printers can also print black and white. How many ways are there for the 125 pages to be assigned to the five printers?\\\\
%Enter your answer below this comment line.
--------------------------------------------------------------------------------------------------\\
After making sure that each printer gets at least one page, leaving 120 pages left over to distribute among the 5 printers. The way that I solved this was by using the "stars and bars" concept. Imagining the 120 pages as stars (*) and the dividers between different printers as bars (|). To distribute all 120 unique pages among 5 printers, I placed 4 bars to create 5 sections within 120 items.\\
The formula for this calculation is \(\binom{n + k - 1}{k - 1}\), where \(n\) is the number of items to distribute (120 pages) and \(k\) is the number of groups (5 printers). The formula calculates the number of combinations of placing \(k - 1\) bars among \(n + k - 1\) slots.\\\\
This makes the calculation: \(\binom{120 + 5 - 1}{5 - 1}\) or \(\binom{124}{4}\).\\
--------------------------------------------------------------------------------------------------\\
        \\\\
 \item Suppose that all the pages are black and white, but each group of 25 consecutive pages (1-25, 26-50, 51-75, 76-100, 101-125) must be assigned to the same printer. Each printer can be assigned 0, 25, 50, 75, 100, or 125 pages to print.\\\\
How many ways are there for the 125 pages to be assigned to the five printers?\\\\
%Enter your answer below this comment line.
--------------------------------------------------------------------------------------------------\\
Since each of the groupings can be assigned to any of the printers, and with the pages grouped as (1-25, 26-50, 51-75, 76-100, 101-125), and with the distinction among them of the printers, in this case, the total number of ways these groups can be assigned to the printers is by calculating the following:\\
Since every group of pages has 5 choices of printers (without any restriction of how many groups a single printer can receive), while there are 5 of these groups, the total number of ways to assign these groups to the printers is \(5^5\).\\
I calculated this based on the principle that for every group there are 5 options (any of the 5 printers), and since the decisioning for one group is independent of the others, it's multiplied. Which means the total number of ways of assigning the 125 pages, which must be partitioned into 5 groups of 25 pages each and assigned to the 5 printers, is \(5^5\).\\
--------------------------------------------------------------------------------------------------\\
        \\\\
   \end{enumerate}
 \newpage
%--------------------------------------------------------------------------------------------------

\section*{Problem 2}
Ten kids line up for recess. The names of the kids are:\\
\begin{center}
 \{Alex, Bobby, Cathy, Dave, Emy, Frank, George, Homa, Ian, Jim\}.\\
\end{center}
Let $S$ be the set of all possible ways to line up the kids. For example, one order might be:
\begin{center}
  (Frank, George, Homa, Jim, Alex, Dave, Cathy, Emy, Ian, Bobby)\\
\end{center}
The names are listed in order from left to right, so Frank is at the front of the line and Bobby is at the end of the line.\\
Let $T$ be the set of all possible ways to line up the kids in which George is ahead of Dave in the line. Note that George does not have to be immediately ahead of Dave. For example, the ordering shown above is an element in $T$.\\
Now define a function $f$ whose domain is $S$ and whose target is $T$. Let $x$ be an element of $S$, so $x$ is one possible way to order the kids. If George is ahead of Dave in the ordering $x$, then $f(x) = x$. If Dave is ahead of George in $x$, then $f(x)$ is the ordering that is the same as $x$, except that Dave and George have swapped places.\\
\begin{enumerate}[label=(\alph*)]
  \item What is the output of $f$ on the following input?\\
  (Frank, George, Homa, Jim, Alex, Dave, Cathy, Emy, Ian, Bobby)\\\\
%Enter your answer below this comment line.
--------------------------------------------------------------------------------------------------\\
The output is the same because George is ahead of Dave.\\ Therefore, $f(x) = x$.\\
--------------------------------------------------------------------------------------------------\\
\\\\
  \item What is the output of $f$ on the following input?\\
(Emy, Ian, Dave, Homa, Jim, Alex, Bobby, Frank, George, Cathy)\\\\
%Enter your answer below this comment line.
--------------------------------------------------------------------------------------------------\\
The output of $f$ is (Emy, Ian, George, Homa, Jim, Alex, Bobby, Frank, Dave, Cathy) because Dave is ahead of George, so their positions are swapped to maintain the condition of $T$.\\
--------------------------------------------------------------------------------------------------\\
\\\\\
  \item Is the function $f$ a $k$-to-1 correspondence for some positive integer $k$? If so, for what value of $k$? Justify your answer.\\\\
%Enter your answer below this comment line.
--------------------------------------------------------------------------------------------------\\
The function $f$ is a $k$-to-1 correspondence for $k = 2$.\\
This is because for any arrangement of the kids, there are only two possibilities regarding the positions of George and Dave: either George is ahead of Dave, or Dave is ahead of George. If George is already ahead, it maps directly to an element in $T$.\\
If Dave is ahead, swapping them also results in an element in $T$. There fore, for every two arrangements in $S$, there is exactly one corresponding arrangement in $T$.\\
--------------------------------------------------------------------------------------------------\\
\\\\
  \item There are 3628800 ways to line up the 10 kids with no restrictions on who comes before whom. That is, $|S| =3628800$. Use this fact and the answer to the previous question to determine $|T|$.\\\\
%Enter your answer below this comment line.
--------------------------------------------------------------------------------------------------\\
Considering that $|S| = 3628800$, and since $f$ is a 2-to-1 correspondence, $|T| = |S| / 2 = 1814400$. This means there are 1,814,400 ways to line up the 10 kids with the constraint that George must be ahead of Dave.\\
--------------------------------------------------------------------------------------------------\\
\\\\
\end{enumerate}
   \newpage
%--------------------------------------------------------------------------------------------------

\section*{Problem 3}
   
   
Consider the following definitions for sets of characters:
\begin{itemize}
  \item Digits $\;=\; \{ 0,\, 1,\, 2,\, 3,\, 4,\, 5,\, 6,\, 7,\, 8,\, 9 \}$\\
  \item Letters$\; = \;\{ a,\, b,\, c, \,d,\, e,\, f,\, g,\, h,\, i,\, j,\, k,\, l,\, m,\, n,\, o,\, p,\, q,\, r,\, s,\, t,\, u,\, v,\, w,\, x,\, y,\, z \}$\\
  \item Special characters $\;=\; \{ *,\, \&,\, \$,\, \# \}$\\
\end{itemize}

Compute the number of passwords that satisfy the given constraints.
    \begin{enumerate}[label=(\roman*)]
    \item Strings of length 7. Characters can be special characters, digits, or letters, with no repeated characters.\\\\
%Enter your answer below this comment line.
--------------------------------------------------------------------------------------------------\\
The total number of characters available is \(40\) (10 digits, 26 letters and 4 special characters). The number of unique passwords among all possible passwords of length 7 where no character can be repeated, is given by the permutation formula \(P(40, 7)\). This represents the number of arrangements of 7 out of the 40 distinct characters.\\
--------------------------------------------------------------------------------------------------\\
\\\\
    \item Strings of length 6 Characters can be special characters, digits, or letters, with no repeated characters. The first character can not be a special character.\\\\
%Enter your answer below this comment line.
--------------------------------------------------------------------------------------------------\\
Strings of length 6 Characters can be special characters, digits, or letters, and no repetition is allowed. The first character cannot be a special character. For the first character, there are \(36\) possible options (which includes only digits and letters but not the special characters). After the selection of the first character, for the next 5 positions, there will be \(39\) characters to select from. The number of unique passwords would then be \(36 \times P(39, 5)\) where \(P(39, 5)\) represents the number of permutations of 5 from the remaining 39 characters.\\
--------------------------------------------------------------------------------------------------\\
      \end{enumerate}
 \newpage
%--------------------------------------------------------------------------------------------------

\section*{Problem 4}
A group of four friends goes to a restaurant for dinner. The restaurant offers 12 different main dishes.\\
    \begin{enumerate}[label=(\roman*)]
    \item Suppose that the group collectively orders four different dishes to share. The waiter just needs to place all four dishes in the center of the table. How many different possible orders are there for the group?\\\\
%Enter your answer below this comment line.
--------------------------------------------------------------------------------------------------\\
The number of different potential orders is given by the combination formula \(\binom{12}{4}\), which calculates how many ways 4 different dishes can be chosen from 12 options to share.\\
--------------------------------------------------------------------------------------------------\\
    \item Suppose that each individual orders a main course. The waiter must remember who ordered which dish as part of the order. It's possible for more than one person to order the same dish. How many different possible orders are there for the group?\\\\
%Enter your answer below this comment line.
--------------------------------------------------------------------------------------------------\\
With each of the four friends able to choose any of the 12 dishes, allowing for repetition, the total number of different possible orders is \(12^4\).\\
--------------------------------------------------------------------------------------------------\\
    \end{enumerate}

How many different passwords are there that contain only digits and lower-case letters and satisfy the given restrictions?\\
      \begin{enumerate}[label=(\roman*), start=3]
    \item Length is 7 and the password must contain at least one digit.\\\\
%Enter your answer below this comment line.
--------------------------------------------------------------------------------------------------\\
The total number of such passwords is calculated as \(36^7 - 26^7\), where \(36^7\) represents all possible 7-character passwords using digits and letters, and \(26^7\) is subtracted to exclude passwords that contain only letters, ensuring at least one digit is present.\\
--------------------------------------------------------------------------------------------------\\
     \item Length is 7 and the password must contain at least one digit and at least one letter.\\\\
%Enter your answer below this comment line.
--------------------------------------------------------------------------------------------------\\
This involves subtracting the number of passwords that are all digits (\(10^7\)) or all letters (\(26^7\)) from the total possible combinations (\(36^7\)), and adjusting for the inclusion of at least one of each type. The exact number would be represented by \(36^7 - 10^7 - 26^7\), ensuring the presence of at least one digit and one letter.\\
--------------------------------------------------------------------------------------------------\\
\\\\
    \end{enumerate}
 
 \newpage
%--------------------------------------------------------------------------------------------------

\section*{Problem 5}

A university offers a Calculus class, a Sociology class, and a Spanish class. You are given data below about two groups of students.\\\\
     \begin{enumerate}[label=(\roman*)]
     \item Group 1 contains 170 students, all of whom have taken at least one of the three courses listed above. Of these, 61 students have taken Calculus, 78 have taken Sociology, and 72 have taken Spanish. 15 have taken both Calculus and Sociology, 20 have taken both Calculus and Spanish, and 13 have taken both Sociology and Spanish. How many students have taken all three classes?\\\\
%Enter your answer below this comment line.
--------------------------------------------------------------------------------------------------\\
Group 1 contains 170 students, all of whom have taken at least one of the three courses listed above. Of these, 61 students have taken Calculus, 78 have taken Sociology, and 72 have taken Spanish. 15 have taken both Calculus and Sociology, 20 have taken both Calculus and Spanish, and 13 have taken both Sociology and Spanish. To find how many students have taken all three classes, we apply the principle of inclusion-exclusion. The correct application involves solving for the number of students who are part of all three sets, using the formula for the principle of inclusion-exclusion.\\
--------------------------------------------------------------------------------------------------\\
\\\\\
\item You are given the following data about Group 2. 32 students have taken Calculus, 22 have taken Sociology, and 16 have taken Spanish. 10 have taken both Calculus and Sociology, 8 have taken both Calculus and Spanish, and 11 have taken both Sociology and Spanish. 5 students have taken all three courses while 15 students have taken none of the courses. How many students are in Group 2?\\\\
%Enter your answer below this comment line.
--------------------------------------------------------------------------------------------------\\
Formula for the principle of inclusion-exclusion for three sets:\\
\[ |C \cup S \cup Sp| = |C| + |S| + |Sp| - |C \cap S| - |C \cap Sp| - |S \cap Sp| + |C \cap S \cap Sp| \]\\
total number of students who have taken at least one course:\\
\[ |C \cup S \cup Sp| = 32 + 22 + 16 - 10 - 8 - 11 + 5 \]
\[ |C \cup S \cup Sp| = 70 - 29 + 5 \]
\[ |C \cup S \cup Sp| = 46 \]\\
Then I added the students who have taken at least one course and those who haven't taken any courses:\\
\[ \text{Total students in Group 2} = |C \cup S \cup Sp| + \text{Students who have taken none of the courses} \]
\[ \text{Total students in Group 2} = 46 + 15 \]
\[ \text{Total students in Group 2} = 61 \]\\
Therefore, there are 61 students in Group 2.\\
--------------------------------------------------------------------------------------------------\\
\\\\\
         \end{enumerate}
 \newpage
%--------------------------------------------------------------------------------------------------

\section*{Problem 6}
A coin is flipped five times. For each of the events described below, express the event as a set in roster notation. Each outcome is written as a string of length 5 from $\{H,\, T\}$, such as $HHHTH$. Assuming the coin is a fair coin, give the probability of each event.\\
\begin{enumerate}[label=(\alph*)]
\item The first and last flips come up heads.\\\\\
%Enter your answer below this comment line.
--------------------------------------------------------------------------------------------------\\
These outcomes can be represented as $H---H$, where each dash can independently be $H$ or $T$. Thus, there are $2^3 = 8$ possible outcomes for this event: $HHHHH, HHHTH, HHTHH, HHTTH, HTHHH, HTHTH,\\ HTHTH, HTTHH$.\\\\
\textbf{Probability}: Since there are 8 favorable outcomes out of a total of 32, the probability of this event is $\frac{8}{32} = \frac{1}{4}$.\\
--------------------------------------------------------------------------------------------------\\
\\\\\
\item There are at least two consecutive flips that come up heads.\\\\\
%Enter your answer below this comment line.
--------------------------------------------------------------------------------------------------\\
This is calculated based on the principle of inclusion-exclusion and considering all possible patterns that meet this criteria. The number of sequences that include at least one pair of consecutive heads is determined to be 78.\\\\
\textbf{Probability}: Given that there are a total of 32 possible outcomes for 5 coin flips, the probability of having at least two consecutive flips that come up heads is $\frac{78}{32}$.\\
--------------------------------------------------------------------------------------------------\\
\\\\\
\item The first flip comes up tails and there are at least two consecutive flips that come up heads.\\\\\
%Enter your answer below this comment line.
--------------------------------------------------------------------------------------------------\\
Given the requirement for the first flip to come up tails (T) and for there to be at least two consecutive flips that come up heads (HH) in the subsequent flips, we analyze the possible outcomes within this framework. The initial flip is fixed as T, reducing our focus to the patterns formed by the remaining four flips. 

The key is to ensure at least one instance of HH occurs within these four positions, allowing for various combinations of H and T, provided the HH requirement is met. Although a detailed enumeration would precisely determine the count of valid sequences, the essence of the calculation lies in identifying and summing the possibilities that conform to these criteria.


\textbf{Probability}: Given the total of 32 possible outcomes for 5 coin flips, the probability of this specific event occurring is $\frac{26}{32}$.

--------------------------------------------------------------------------------------------------\\
\\\\\
\end{enumerate}

 \newpage
%--------------------------------------------------------------------------------------------------

\section*{Problem 7}
An editor has a stack of $k$ documents to review.  The order in which the documents are reviewed is random with each ordering being equally likely. Of the $k$ documents to review, two are named ``Relaxation Through Mathematics'' and ``The Joy of Calculus.'' Give an expression for each of the probabilities below as a function of k. Simplify your final expression as much as possible so that your answer does not include any expressions in the form\\\\
$
\Big(
 \begin{array}{c}
 a\\
 b
    \end{array}
    \Big)
$.
\\\\
 \begin{enumerate}[label=(\alph*)]
\item What is the probability that `Relaxation Through Mathematics' is first to review?\\\\
%Enter your answer below this comment line.
--------------------------------------------------------------------------------------------------\\
\[\text{Probability} = \frac{1}{k!}\]
--------------------------------------------------------------------------------------------------\\
\\\\
\item What is the probability that `Relaxation Through Mathematics' and `The Joy of Calculus' are next to each other in the stack?\\\\
%Enter your answer below this comment line.
--------------------------------------------------------------------------------------------------\\
\[\text{Probability} = \frac{2}{k \cdot (k - 1)}\]\\
--------------------------------------------------------------------------------------------------\\
\\\\
\end{enumerate}


\end{document}
