
Based on the displayed Hasse diagram, here are the answers to the questions:

\begin{enumerate}
    \item[(a)] The \textbf{minimal elements} are those that have no edges descending to them. In this diagram, the minimal element is:
    \begin{itemize}
        \item \textbf{A}.
    \end{itemize}
    
    \item[(b)] The \textbf{maximal elements} are those with no edges ascending from them. In this diagram, the maximal elements are:
    \begin{itemize}
        \item \textbf{J}, \textbf{D}, and \textbf{G}.
    \end{itemize}
    
    \item[(c)] Two elements are \textbf{comparable} if you can travel from one to the other following the edges without backtracking. Based on this diagram:
    \begin{itemize}
        \item \textbf{A} and \textbf{B} are comparable because you can move upwards from \textbf{A} to \textbf{B}.
        \item \textbf{B} and \textbf{C} are comparable because you can move upwards from \textbf{B} to \textbf{C}.
        \item \textbf{C} and \textbf{D} are comparable because you can move upwards from \textbf{C} to \textbf{D}.
        \item \textbf{D} and \textbf{E} are not comparable; there's no way to move from \textbf{D} to \textbf{E} or vice versa following the edges.
        \item \textbf{E} and \textbf{F} are comparable because you can move downwards from \textbf{E} to \textbf{F}.
        \item \textbf{E} and \textbf{G} are comparable because you can move upwards from \textbf{E} to \textbf{G}.
    \end{itemize}
\end{enumerate}
